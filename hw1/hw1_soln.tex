%%%%%%%%%%%%%%%%%%%%%%%%%%%%%%%%%%%%%%%%%
% Structured General Purpose Assignment
% LaTeX Template
%
% This template has been downloaded from:
% http://www.latextemplates.com
%
% Original author:
% Ted Pavlic (http://www.tedpavlic.com)
%
% Note:
% The \lipsum[#] commands throughout this template generate dummy text
% to fill the template out. These commands should all be removed when 
% writing assignment content.
%
%%%%%%%%%%%%%%%%%%%%%%%%%%%%%%%%%%%%%%%%%

%----------------------------------------------------------------------------------------
%	PACKAGES AND OTHER DOCUMENT CONFIGURATIONS
%----------------------------------------------------------------------------------------

\documentclass{article}

\usepackage{fancyhdr} % Required for custom headers
\usepackage{lastpage} % Required to determine the last page for the footer
\usepackage{extramarks} % Required for headers and footers
\usepackage{graphicx} % Required to insert images
\usepackage{lipsum} % Used for inserting dummy 'Lorem ipsum' text into the template
\usepackage{amsfonts}
\usepackage{subfig}
\usepackage{graphicx}
\usepackage{caption}
\usepackage{amsmath}

%
%% Margins
%\topmargin=-0.45in
%\evensidemargin=0in
%\oddsidemargin=0in
%\textwidth=6.5in
%\textheight=9.0in
%\headsep=0.25in 
%
%\linespread{1.1} % Line spacing
%
%% Set up the header and footer
%\pagestyle{fancy}
%\lhead{\hmwkAuthorName} % Top left header
%\chead{\hmwkClass\ (\hmwkClassInstructor): \hmwkTitle} % Top center header
%\rhead{\firstxmark} % Top right header
%\lfoot{\lastxmark} % Bottom left footer
%\cfoot{} % Bottom center footer
%\rfoot{Page\ \thepage\ of\ \pageref{LastPage}} % Bottom right footer
%\renewcommand\headrulewidth{0.4pt} % Size of the header rule
%\renewcommand\footrulewidth{0.4pt} % Size of the footer rule
%
%\setlength\parindent{0pt} % Removes all indentation from paragraphs
%
%%----------------------------------------------------------------------------------------
%%	DOCUMENT STRUCTURE COMMANDS
%%	Skip this unless you know what you're doing
%%----------------------------------------------------------------------------------------
%
%% Header and footer for when a page split occurs within a problem environment
%\newcommand{\enterProblemHeader}[1]{
%\nobreak\extramarks{#1}{#1 continued on next page\ldots}\nobreak
%\nobreak\extramarks{#1 (continued)}{#1 continued on next page\ldots}\nobreak
%}
%
%% Header and footer for when a page split occurs between problem environments
%\newcommand{\exitProblemHeader}[1]{
%\nobreak\extramarks{#1 (continued)}{#1 continued on next page\ldots}\nobreak
%\nobreak\extramarks{#1}{}\nobreak
%}
%
%\setcounter{secnumdepth}{0} % Removes default section numbers
%\newcounter{homeworkProblemCounter} % Creates a counter to keep track of the number of problems
%
%\newcommand{\homeworkProblemName}{}
%\newenvironment{homeworkProblem}[1][Problem \arabic{homeworkProblemCounter}]{ % Makes a new environment called homeworkProblem which takes 1 argument (custom name) but the default is "Problem #"
%\stepcounter{homeworkProblemCounter} % Increase counter for number of problems
%\renewcommand{\homeworkProblemName}{#1} % Assign \homeworkProblemName the name of the problem
%\section{\homeworkProblemName} % Make a section in the document with the custom problem count
%\enterProblemHeader{\homeworkProblemName} % Header and footer within the environment
%}{
%\exitProblemHeader{\homeworkProblemName} % Header and footer after the environment
%}
%
%\newcommand{\problemAnswer}[1]{ % Defines the problem answer command with the content as the only argument
%\noindent\framebox[\columnwidth][c]{\begin{minipage}{0.98\columnwidth}#1\end{minipage}} % Makes the box around the problem answer and puts the content inside
%}
%
%\newcommand{\homeworkSectionName}{}
%\newenvironment{homeworkSection}[1]{ % New environment for sections within homework problems, takes 1 argument - the name of the section
%\renewcommand{\homeworkSectionName}{#1} % Assign \homeworkSectionName to the name of the section from the environment argument
%\subsection{\homeworkSectionName} % Make a subsection with the custom name of the subsection
%\enterProblemHeader{\homeworkProblemName\ [\homeworkSectionName]} % Header and footer within the environment
%}{
%\enterProblemHeader{\homeworkProblemName} % Header and footer after the environment
%}
   
%----------------------------------------------------------------------------------------
%	NAME AND CLASS SECTION
%----------------------------------------------------------------------------------------

\newcommand{\hmwkTitle}{Assignment\ \#1} % Assignment title
\newcommand{\hmwkDueDate}{Thursday,\ Jaunary\ 14,\ 2016} % Due date
\newcommand{\hmwkClass}{CS\ 260} % Course/class
\newcommand{\hmwkClassInstructor}{Prof. Wim van Dam} % Teacher/lecturer
\newcommand{\hmwkAuthorName}{Chad Spensky} % Your name

%----------------------------------------------------------------------------------------
%	TITLE PAGE
%----------------------------------------------------------------------------------------

\title{
\vspace{2in}
\textmd{\textbf{\hmwkClass:\ \hmwkTitle}}\\
\normalsize\vspace{0.1in}\small{Due\ on\ \hmwkDueDate}\\
\vspace{0.1in}\large{\textit{\hmwkClassInstructor}}
\vspace{3in}
}

\author{\textbf{\hmwkAuthorName}}
\date{} % Insert date here if you want it to appear below your name

%----------------------------------------------------------------------------------------

\begin{document}

\maketitle
\newpage

\input{macros}

%----------------------------------------------------------------------------------------
%	TABLE OF CONTENTS
%----------------------------------------------------------------------------------------

%\setcounter{tocdepth}{1} % Uncomment this line if you don't want subsections listed in the ToC
%
%\newpage
%\tableofcontents
%\newpage

%----------------------------------------------------------------------------------------
%	PROBLEM 1
%----------------------------------------------------------------------------------------

% To have just one problem per page, simply put a \clearpage after each problem

\section{Question 1}
The problem asks to prove that:
\begin{equation*}
H(X) \leq 2log(\sum_{x \in \fancyx{}} \sqrt{p(x)}
\end{equation*}
This can be done in the following way:
\begin{align}
H(X) & = \sum_{x \in \fancyx} -p(x)log(p(x))\\
& =  \sum_{x \in \fancyx} -2\frac{1}{2}p(x)log(p(x))\\
& =  \sum_{x \in \fancyx} 2p(x)log(\frac{1}{\sqrt{p(x)} })\\
& =  E_{p} [2 log(\frac{1}{ \sqrt{p(x)} })) ]\\
& \leq  2 log[ E_{p} (\frac{1}{ \sqrt{p(x)} }) ]\label{one_jensens}\\
2 log[ E_{p} (\frac{1}{ \sqrt{p(x)} }) ] & =  2 log[ \sum_{x \in \fancyx} \frac{p(x)}{ \sqrt{p(x)} } ]\\
& =  2 log[ \sum_{x \in \fancyx} \sqrt{p(x)} ]
\end{align}
where \ref{one_jensens} is an applications of Jensen's inequality. \QED

\newpage

\section{Question 2}
In this question we are asked for the various entropies give all of the joint probabilities.
\begin{align*}
Pr[\fancyx = 0, \fancyy = a] = 0.15 &~~~
Pr[\fancyx = 0, \fancyy = b] = 0.3 &
Pr[\fancyx = 0, \fancyy = c] = 0.05\\
Pr[\fancyx = 1, \fancyy = a] = 0.25 &~~~
Pr[\fancyx = 1, \fancyy = b] = 0.15 &
Pr[\fancyx = 1, \fancyy = c] = 0.1\\
\end{align*}
First, we'll define the individual probability distributions:
\begin{align*}
Pr[\fancyx = 0] &= Pr[\fancyx = 0, \fancyy = a] + Pr[\fancyx = 0, \fancyy = b]  + Pr[\fancyx = 0, \fancyy = c]  &= 0.5\\
Pr[\fancyx = 1] &= Pr[\fancyx = 1, \fancyy = a] + Pr[\fancyx = 1, \fancyy = b]  + Pr[\fancyx = 1, \fancyy = c]  &= 0.5\\
\\
Pr[\fancyy = a] &= Pr[\fancyx = 0, \fancyy = a]  + Pr[\fancyx = 1, \fancyy = a] &= 0.4\\
Pr[\fancyy = b] &= Pr[\fancyx = 0, \fancyy = b]  + Pr[\fancyx = 1, \fancyy = b] &= 0.45\\
Pr[\fancyy = c] &= Pr[\fancyx = 0, \fancyy = c]  + Pr[\fancyx = 1, \fancyy = c] &= 0.15
\end{align*}
First, we will calculate the basic entropies:
\begin{align*}
H(X) &= \sum_{x \in \fancyx} -p(x)log(p(x))\\
 &= -0.5*log(0.5)-0.5*log(0.5) &= 0.301\\
H(Y) &= \sum_{y \in \fancyy} -p(y)log(p(y))\\
 &= -0.4*log(0.4)-0.45*log(0.45)-0.15*log(0.15) &= 0.439
\end{align*}
Recall:
\begin{align*}
Pr[X,Y] = Pr[X]*Pr[Y|X]\\
Pr[Y|X] = \frac{Pr[X,Y]}{Pr[Y]}
\end{align*}

Next, we will compute the conditional entropies:
\begin{align*}
H(X | Y = a) &= \sum_{x \in \fancyx} -p(x|y=a)log(p( x | y=a))\\
&= -\frac{0.15}{0.5}log(\frac{0.15}{0.5}) - \frac{0.25}{0.5} log(\frac{0.25}{0.5})\\
&= 0.307\\
\\
H(X | Y = b) &= \sum_{x \in \fancyx} -p(x|y=b)log(p( x | y=b))\\
&= -\frac{0.3}{0.5}log(\frac{0.3}{0.5}) - \frac{0.15}{0.5} log(\frac{0.15}{0.5})\\
&= 0.290\\
\\
H(X | Y = c) &= \sum_{x \in \fancyx} -p(x|y=c)log(p( x | y=c))\\
&= -\frac{0.05}{0.5}log(\frac{0.05}{0.5}) - \frac{0.1}{0.5} log(\frac{0.1}{0.5})\\
&= 0.240\\
\\
H(X|Y) &= \sum_{y \in \fancyy} p(y) H(X|Y=y)\\
&= 0.4*0.307 + 0.45*0.29 + 0.15*0.240\\
&= .289
\\
\end{align*}
or, more concisely (ignoring rounding errors):
\begin{align*}
H(X | Y) =& \sum_{x \in \fancyx ,y \in \fancyy} p(x,y)log(\frac{p(y)}{p( x, y)})\\
=&~0.15log(\frac{0.4}{0.15}) + 0.3 log(\frac{0.45}{0.3}) + 0.05 log(\frac{0.15}{0.05})\\
&+ 0.25log(\frac{0.4}{0.25}) + 0.15 log(\frac{0.45}{0.15}) + 0.1 log(\frac{0.15}{0.1})\\
=&~0.281
\\
H(Y | X) =& \sum_{x \in \fancyx ,y \in \fancyy} p(x,y)log(\frac{p(x)}{p( x, y)})\\
=&~0.15 log(\frac{0.5}{0.15}) + 0.3 log(\frac{0.5}{0.3}) + 0.05 log(\frac{0.5}{0.05})\\
&+ 0.25 log(\frac{0.5}{0.25}) + 0.15 log(\frac{0.5}{0.15}) + 0.1 log(\frac{0.5}{0.1})\\
=&~0.419
% &= -0.5*log(0.5)-0.5*log(0.5) &= 0.301\\
%H(Y) &= \sum_{y \in \fancyy} -p(y)log(p(y))\\
% &= -0.4*log(0.4)-0.45*log(0.45)-0.15*log(0.15) &= 0.439
\end{align*}
Thus, we obtain:
\begin{align*}
H(X,Y) &= H(X) + H(Y|X)\\
&= 0.301+0.419\\
& = 0.720\\
H(X,Y) &= H(Y) + H(X|Y)\\
&= 0.439+0.281\\
&= 0.720\\
\end{align*}
Finally, we compute:
\begin{align*}
I(X;Y) &= H(X) - H(X|Y)\\
&= 0.301 - 0.281\\
&= 0.02\\
I(Y;X) &= H(Y) - H(Y|X)\\
&= 0.439 - 0.419\\
&= 0.02\\
\end{align*}
\QED
\newpage

\section{Question 3}
In this question we given the distribution $p_{\lambda} = (1-\lambda)*p_0 + \lambda*p_1$, asked to prove relationships between $H(p_{\lambda}), H(p_0), H(p_1), \lambda,$ and $D$.

The immediately obvious ones are:
\begin{align*}
H(p_{\lambda}) &= H(p_0) & (\lambda = 0)\\
H(p_{\lambda}) &= H(p_1) & (\lambda = 1)\\
H(p_{\lambda}) &\leq H(p_0) + H(p_1) \\
\end{align*}
\begin{align*}
D(p||q) = D(p||q) = 0 \Rightarrow p = q \Rightarrow H(p_{\lambda}) = H(p_0) = H(p_1)
\end{align*}
Intuitively, it seems like something of the form
\begin{align*}
H(p_{\lambda}) &= (1-\lambda)*H(p_0) + \lambda*H(p_1)\\
\end{align*}
Note that
\begin{align*}
p_0 &= \frac{p_\lambda - \lambda p_1}{1- \lambda}\\
p_1 &= \frac{p_\lambda - (1-\lambda)p_0}{\lambda}\\
\end{align*}
Thus
\begin{align*}
H(p_1) &= \sum_{x \in \fancyx} \frac{p_\lambda - (1-\lambda)p_0}{\lambda} log(\frac{p_\lambda - (1-\lambda)p_0}{\lambda})\\
&= \sum_{x \in \fancyx} \frac{p_\lambda - (1-\lambda)p_0}{\lambda} log(p_\lambda - (1-\lambda)p_0) - \frac{p_\lambda - (1-\lambda)p_0}{\lambda} log(\lambda)\\
&= \sum_{x \in \fancyx} \frac{p_\lambda}{\lambda} log(p_\lambda - (1-\lambda)p_0) 
 - \frac{1}{\lambda} log(p_\lambda - (1-\lambda)p_0) 
  + \frac{\lambda p_0}{\lambda} log(p_\lambda - (1-\lambda)p_0) - \frac{p_\lambda - (1-\lambda)p_0}{\lambda} log(\lambda)\\
  &= \sum_{x \in \fancyx} \frac{p_\lambda}{\lambda} log(\lambda p_1) 
 - \frac{1}{\lambda} log(\lambda p_1) 
  + \frac{\lambda p_0}{\lambda} log(\lambda p_1) - \frac{p_\lambda - (1-\lambda)p_0}{\lambda} log(\lambda)\\
&= \sum_{x \in \fancyx} \frac{(1-\lambda)p_0 + \lambda p_1}{\lambda} log(\lambda p_1) 
 - \frac{1}{\lambda} log(\lambda p_1) 
  + \frac{\lambda p_0}{\lambda} log(\lambda p_1) - \frac{p_\lambda - (1-\lambda)p_0}{\lambda} log(\lambda)\\
  &= \sum_{x \in \fancyx} \frac{(1-\lambda)p_0}{\lambda} log(\lambda p_1) + p_1  log(\lambda p_1)
 - \frac{1}{\lambda} log(\lambda p_1) 
  + \frac{\lambda p_0}{\lambda} log(\lambda p_1) - \frac{p_\lambda - (1-\lambda)p_0}{\lambda} log(\lambda)\\
    \\
H(p_0) &= \sum_{x \in \fancyx} \frac{p_\lambda - \lambda p_1}{1- \lambda} log(\frac{p_\lambda - \lambda p_1}{1- \lambda})\\
&= \sum_{x \in \fancyx} \frac{p_\lambda - \lambda p_1}{1- \lambda} log(p_\lambda - \lambda p_1) - \frac{p_\lambda - \lambda p_1}{1- \lambda} log(1- \lambda)\\
&= \sum_{x \in \fancyx} \frac{p_\lambda}{1- \lambda} log(p_\lambda - \lambda p_1) - \frac{\lambda p_1}{1- \lambda}log(p_\lambda - \lambda p_1) - \frac{p_\lambda - \lambda p_1}{1- \lambda} log(1- \lambda)\\
&= \sum_{x \in \fancyx} \frac{p_\lambda}{1- \lambda} log((1 - \lambda) p_0) - \frac{\lambda p_1}{1- \lambda}log((1- \lambda) p_0) - \frac{p_\lambda - \lambda p_0}{1- \lambda} log(1- \lambda)\\
&= \sum_{x \in \fancyx} \frac{(1-\lambda) p_0 + \lambda p_1}{1- \lambda} log((1 - \lambda) p_0) - \frac{\lambda p_1}{1- \lambda}log((1- \lambda) p_0) - \frac{p_\lambda - \lambda p_0}{1- \lambda} log(1- \lambda)\\
&= \sum_{x \in \fancyx} p_0 log((1 - \lambda) p_0) + \frac{\lambda p_1}{1- \lambda} log((1 - \lambda) p_0) - \frac{\lambda p_1}{1- \lambda}log((1- \lambda) p_0) - \frac{p_\lambda - \lambda p_0}{1- \lambda} log(1- \lambda)\\
\\
H(p_\lambda) &= \sum_{x \in \fancyx} ((1-\lambda)p_0 + \lambda p_1) log((1-\lambda) p_0 + \lambda p_1)\\
&= \sum_{x \in \fancyx} (1-\lambda)p_0 log((1-\lambda) p_0 + \lambda p_1)
 + \lambda p_1 log((1-\lambda) p_0 + \lambda p_1)\\
 &= \sum_{x \in \fancyx} log((1-\lambda) p_0 + \lambda p_1) - \lambda p_0 log((1-\lambda) p_0 + \lambda p_1)
 + \lambda p_1 log((1-\lambda) p_0 + \lambda p_1)\\
  &= \sum_{x \in \fancyx} log(p_\lambda) - \lambda p_0 log(p_\lambda) + \lambda p_1 log(p_\lambda)\\
  \\
  H(p_\lambda) -H(p_0) &= \sum_{x \in \fancyx} log(p_\lambda) - \lambda p_0 log(p_\lambda) + \lambda p_1 log(p_\lambda) - p_0log(p_0)\\
  &= \sum_{x \in \fancyx} log(p_\lambda) - p_0(\lambda log(p_\lambda) + log(p_0)) + \lambda p_1 log(p_\lambda) 
\end{align*}
Nevertheless, after countless wasted hours and scratch paper, I am unable to come up with anything more interesting than the obvious w.r.t. these relationships...

\newpage

\section{Question 4}
For this question, we were asked to prove or disprove the following relationship:
\begin{align*}
D(p||q) + D(q||r) \geq D(p||r)
\end{align*}
We can disprove this with a counter example.  First, write out our definitions.
\begin{align*}
D(p||q) &= \sum_{x \in \fancyx} p(x) log(\frac{p(x)}{q(x)})\\
D(q||r) &= \sum_{x \in \fancyx} q(x) log(\frac{q(x)}{r(x)})\\
D(p||r) &= \sum_{x \in \fancyx} p(x) log(\frac{p(x)}{r(x)})\\
\end{align*}
Thus, we would like to show the following:
\begin{align}
\sum_{x \in \fancyx} p(x) log(\frac{p(x)}{q(x)}) +  \sum_{x \in \fancyx} q(x) log(\frac{q(x)}{r(x)}) &\geq  \sum_{x \in \fancyx} p(x) log(\frac{p(x)}{r(x)})\\
\sum_{x \in \fancyx} p(x) log(\frac{p(x)}{q(x)}) -  \sum_{x \in \fancyx} p(x) log(\frac{p(x)}{r(x)}) &\geq \sum_{x \in \fancyx} q(x) log(\frac{q(x)}{r(x)})\\
\sum_{x \in \fancyx} p(x) [ log(\frac{p(x)}{q(x)}) - log(\frac{p(x)}{r(x)}) ] &\geq \sum_{x \in \fancyx} q(x) log(\frac{q(x)}{r(x)})\\
\sum_{x \in \fancyx} p(x) log(\frac{r(x)}{q(x)}) &\geq \sum_{x \in \fancyx} q(x) log(\frac{q(x)}{r(x)})\\
\sum_{x \in \fancyx} [ p(x) log(\frac{r(x)}{q(x)}) - q(x) log(\frac{q(x)}{r(x)}) ] &\geq 0
\end{align}
If we assume that $p(x) = q(x)$, this reduces to:
\begin{align}
\sum_{x \in \fancyx} p(x) [log(\frac{r(x)}{p(x)} - log(\frac{p(x)}{r(x)})] &\geq 0\\
\sum_{x \in \fancyx} p(x) log(\frac{r(x)^2}{p(x)^2}) &\geq 0\\
\sum_{x \in \fancyx} p(x) log[ (\frac{r(x)}{p(x)})^2 ] &\geq 0\\
\sum_{x \in \fancyx} 2p(x) log(\frac{r(x)}{p(x)}) &\geq 0\\
\end{align}
If we let $\fancyx = \{0,1\}$ s.t. $p(0) = p(1) = .5,$ and $r(0) = .7, r(1) = .3)$, then:
\begin{align*}
\sum_{x \in \fancyx} 2p(x) log(\frac{r(x)}{p(x)}) &= 2*0.5log(\frac{.7}{.5}) + 2*0.5log(\frac{.3}{.5})\\
&= -0.076 < 0
\end{align*}
Which violates our assumption, thus this inequality does not hold in the general case.  \QED


%\end{homeworkProblem}

\end{document}